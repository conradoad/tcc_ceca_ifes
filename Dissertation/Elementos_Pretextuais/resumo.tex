\begin{resumo}

A constante evolução das técnicas e algoritmos de aprendizagem de máquina aplicados na detecção e reconhecimento de faces através do processamento inteligente de imagens tem possibilitado uma gama de aplicações inteligentes e automatizações no monitoramento e controle de acesso a ambientes. E o poder computacional cada vez maior de pequenos dispositivos como celulares, notebooks, SBC's (\textit{Single Board Computers}), câmeras inteligentes, etc., têm possibilitado embarcar o processamento inteligente de imagens nesses dispositivos, ampliando ainda mais a gama de aplicações. No contexto de um sistema de monitoramento com arquitetura de processamento distribuído, esse trabalho traz um estudo sobre a aplicabilidade de se realizar a etapa de detecção de faces em dispositivos de borda em dois diferentes cenários. Para os testes, adotou-se o algoritmo (\textit{Haar Cascade}), conhecidamente rápido na detecção de objetos e adequado para rodar em dispositivos com pouca capacidade de processamento, e um modelo pré-treinado para detecção de faces. Foi desenvolvida uma ferramenta para auxiliar na melhor parametrização do algoritmo e na obtenção dos dados para análise. Por fim, os resultados obtidos foram analisados e feitas a devidas considerações quanto à capacidade da detecção de faces e tempo de resposta que os dispositivos testados desempenharam em cada cenário, bem como também foram abordadas as vantagens em se executar esse processamento na borda, principalmente no que tange à utilização de banda na rede. 

\vspace{0.5cm}
 
 \textbf{Palavras-chave:} Processamento Digital de Imagens; Detecção de Faces; Sistemas de Processamento Distribuído; Processamento Inteligente em Dispositivos de Borda;

\end{resumo}
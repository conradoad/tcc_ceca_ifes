\captionsenglish

\begin{center}
    \bfseries\MakeUppercase{Abstract}
\end{center}

    The constant evolution of machine learning techniques and algorithms applied to face detection and recognition through intelligent image processing has enabled a wide range of intelligent applications and automations in monitoring and access control to environments. The increasing computational power of small devices such as cell phones, notebooks, single board computers (SBC's), smart cameras, etc., has made it possible to embed intelligent image processing on these devices, further expanding the range of applications. In the context of a monitoring system with a distributed processing architecture, this work presents a study on the applicability of performing the face detection step on edge devices in two different scenarios. For the tests, the Haar Cascade algorithm was adopted, which is well-known for its speed in object detection and suitability for running on devices with low processing capacity, along with a pre-trained model for face detection. A tool was developed to assist in the best parameterization of the algorithm and the collection of data for analysis. Finally, the obtained results were analyzed, and proper considerations were made regarding the face detection capability and response time that the tested devices performed in each scenario. Additionally, the advantages of executing this processing at the edge were addressed, especially concerning network bandwidth utilization.
 \vspace{0.5cm}
 
\textbf{Keywords:} Digital Image Processing; Face Detection; Distributed Processing Systems; Intelligent Processing on Edge Devices.
\captionsbrazil
\chapter{Introdução}
\label{chap:intro}
\thispagestyle{plain}

O problema de segurança pública sempre foi um problema sério no Brasil. Constantes crimes de furtos e roubos geram grande danos, principalmente financeiros, tanto para o indivíduo, no caso das vítimas, quanto para a sociedade \cite{Cerqueira2007, G12013}. Um dos reflexos gerados por essa insegurança está no fato de que vários locais públicos devem ter seu acesso controlado e vigiado para garantir a segurança patrimonial.

As \textit{smart cities} (cidades inteligentes) estão emergindo como uma prioridade para pesquisa e desenvolvimento em todo o mundo. Elas abrem oportunidades significativas em várias áreas, como crescimento econômico, saúde, bem-estar, eficiência energética e transporte, para promover o desenvolvimento sustentável das cidades \cite{Song2017}. O conceito e oportunidades das \textit{smart cities} são escaláveis para outros conceitos ‘smart’ como o \textit{smart room}, \textit{smart home}, \textit{smart building}, etc \cite{Pacheco2018}. A proliferação de tecnologias de informação e comunicação possibilita o desenvolvimento de diversos serviços inteligentes. E um dos serviços comunitários mais essenciais é justamente a vigilância inteligente \cite{Chen2016, Nikouei2018}.

Nos últimos anos, aplicações de reconhecimento facial a partir de imagens geradas por câmeras de videomonitoramento têm ganhado relevância, sendo largamente utilizadas para a verificação ou identificação de indivíduos em locais públicos. No âmbito da segurança, o reconhecimento facial em vídeo permite agilidade nas situações em que muitos indivíduos devem ser identificados rapidamente \cite{Quirita2014}.

Apesar de estarmos longe de conseguir com que a IA (inteligência artificial) se aproxime da performance humana, em algumas áreas, como reconhecimento de imagem, carros autônomos e jogos eletrônicos, ela se mostra equivalente, ou até mesmo superior \cite{Aggarwal2018}. Em tarefas de visão computacional é possível rastrear o movimento de uma pessoa em um plano de fundo complexo. E com moderado sucesso, é possível tentar localizar e nomear todas as pessoas em uma fotografia, através da detecção e reconhecimento de faces, roupas e cabelos \cite{Szeliski2011}.

O \textit{deep learning} (aprendizagem profunda), ramo do \textit{machine learning} (aprendizagem de máquina), tornou-se imensamente popular no reconhecimento de imagens, bem como em outras tarefas de reconhecimento e correspondência de padrões \cite{Verhelst2017}.

As redes neurais artificiais simulam o sistema nervoso humano com base em aprendizado de máquina, tratando as unidades computacionais em um modelo de aprendizado de maneira semelhante aos neurônios humanos. Não é uma tarefa fácil pois o poder computacional do computador mais rápido atualmente equivale a uma pequena fração do poder computacional de um cérebro humano \cite{Aggarwal2018}. 

As \textit{deep neural networks} (redes neurais profundas) envolvem uma complexidade computacional significativa, fazendo com que, até recentemente, seu processamento fosse viável apenas em plataformas de potentes servidores disponíveis na ‘nuvem’ \cite{Verhelst2017}. Quando é necessário armazenamento e computação de dados em larga escala, a computação em nuvem tem sido a solução. Porém, com o grande crescimento de dispositivos móveis e inteligentes, juntamente com as tecnologia de IoT (Internet das Coisas), o foco mudou para se obter respostas em tempo real. \cite{Dolui2017}.

Nos últimos anos, vê-se uma tendência de se incorporar o processamento de aprendizado profundo em dispositivos de borda, como celulares, dispositivos móveis e nos nós da IoT. Isso torna possível a análise de dados localmente, em tempo real, além de mitigar problemas de privacidade dos dados \cite{Verhelst2017}. Outro benefício da computação na borda (\textit{Edge Computing}) é o descongestionamento da rede de dados, pois permite que o processamento seja feito próximo das fontes dos dados \cite{Merenda2020}. Assim, evita-se a comunicação desnecessária, que sobrecarrega não só a rede principal como também o datacenter na nuvem \cite{Aazam2014}.

\section{Objetivo Geral}

Com este trabalho objetiva-se testar o desempenho de dispositivos de borda no processo de detecção de faces, avaliando sua capacidade de detecção e tempo de resposta para diferentes cenários, com possíveis aplicações de monitoramento inteligente e controle de acesso. Com os resultados obtidos, espera-se determinar, para cada cenário definido, se o dispositivo é capaz de processar de forma satisfatória a etapa de detecção de faces, e as vantagens de se realizar esse processo na borda, em uma arquitetura de processamento distribuído.

\subsection{Objetivos Específico}
Em um sentido mais estrito, pretende-se
\begin{itemize}
    \item Definir diferentes cenários (com aplicabilidade para monitoramento inteligente e controle de acesso) e os requisitos a serem cumpridos, como tempos de respostas e capacidade de reconhecimento. Serão utilizadas imagens estáticas que representem cada cenário para os testes.
    \item Desenvolver uma ferramenta cliente-servidor para auxiliar na parametrizacão do algoritmo de detecção, buscando melhor otimização para cada cena e dispositivo, e na obtenção das métricas de desempenho.
    \item Analisar os resultados obtidos e determinar para quais cenários os dispositivos podem executar a detecção de face de forma satisfatória e quais os ganhos em se executar tal processamento na borda, principalmente no que se refere à utilização de banda na rede de um sistema com arquitetura de processamento distribuído, sua escalabilidade e tempos de resposta.
\end{itemize}

\section{Estrutura do Texto}
<a preencher>
%Para a exposição do trabalho realizado, este texto foi dividido em cinco capítulos, sendo o Capítulo 1, Introdução, responsável pela apresentação do texto bem como por situar o leitor no contexto ao qual este trabalho se insere. No Capítulo 2, Revisão Teórica, são apresentados alguns conceitos que serão necessário ao bom entendimento das técnicas e soluções propostas. O capítulo 3, Desenvolvimento, explica qual a abordagem utilizada diante do problema, exibindo as diferentes etapas envolvidas na solução do problema, enquanto que o Capítulo 4, Experimentos Realizados, mostra a implementação do sistema proposto. Por fim os resultados obtidos são avaliados e discutidos no Capítulo 5, de nome Conclusão.
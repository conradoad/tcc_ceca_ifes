\chapter{Considerações Finais e Trabalhos \mbox{Futuros}}\label{cap:conclusao}
\thispagestyle{plain}

Esse trablho se propôs a realizar um estudo sobre a aplicabilidade e vantagens da utilização de dispositivos de borda para a etapa de detecção de faces em um contexto de sistemas distribuídos.

Através do desenvolvimento de uma ferramenta para a análise, parametrização e obtenção de dados, foi possível testar a deteção de faces em dois dispositivos SBC (single board computer), o Raspberry Pi 4B e o Rapberry Pi Zero W, utilizando-se o algoritmo \textit{Haar cascade object detection}, bastante difundido e rápido, adequado para rodar em dispositivos com poucos recursos computacionais. Com os dados obtidos a partir das diversas imagens preparadas para representar diferentes cenários e condições de testes foi possível realizar as anáises desejadas quanto ao desempenho no processo de detecção e as vantagens de se executá-lo na borda.

Como resultado geral das análises apresentadas nos capítulos anteriores pôde-se concluir que, com a adequada parametrização do algoritmo de detecção, é possível a utilização de forma satisfatória de dispositivos de borda na detecção de faces para diferentes cenários de aplicação. O que deve-se observar é que para cada cenário haverá um dispositivo mais adequado no que se refere à capacidade computacional e custo. Além disso, a resolução da imagem utilizada na deteção também é fator determinando na qualidade de deteção e tempo de resposta.

Na primeira cena estudada, com a presença de várias faces distantes em um ambiente aberto, apenas o Raspberry Pi 4B apresentou resultados satisfatórios quanto ao tempo de resposta, devido ao seu maior poder computacional. A resolução da imagem nesse caso é fator importante na qualidade da detecção. Utilizando-se imagens com 1440p e 1080p a maioria das faces presentes foram detectadas.

Na segunda cena, onde se tem a detecção de apenas uma face, mas com maior exigência quanto ao tempo de detecção, o Raspberry Pi 4B também apresentou resultados muito satisfatórios, conseguindo realizar a detecção em um tempo consideravelmente menor que o exigido. O Raspberry Pi Zero W também apresentou bons resultados, porém com tempo de detecção um pouco acima do estabelecido para a cena, o que não deve torná-lo inviável, já que com algumas otimizações há grande potencial de redução desse tempo. Além disso, a utilização de um dispositivo de custo mais baixo pode ser suficiente para dar viabilidade financeira a um projeto.

Por fim, ao se comparar os efeitos na utilização de banda, a vantagem de se realizar a detecção de faces no mesmo dispositivo conectado ao sensor de imagem é bastante significativa. Como o dispositivo de borda irá transmitir pela rede apenas os dados referentes às partes das faces detectadas e não a imagem inteira, a economia na utilização de banda que foi observada é de no mínimo 90\%. Assim, além de porporcionar uma economia com infraestrutura de rede, evita-se a necessidade de equipamentos mais potentes que concentrariam a detecção em imagens obtidas de vários pontos de captura, proporcionando também maior escalabilidade.

Como possíveis abordagens em trabalhos futuros, pode-se sugerir:
\begin{itemize}
    \item Avaliar o possível ganho no \textit{throughput}, quando em execução contínua, separando-se a captura da imagem e o algoritmo de detecção em diferentes threads;
    \item Tendo em vista o bom desempenho do Raspberry Pi 4B na dectação de apenas uma face próxima (cena 2), avaliar a capacidade e tempo de resposta do dispositivo em realizar também o reconhecimento da face detectada.
\end{itemize}